%
%  Reporte. Proyecto 2. Análisis forense de redes de computadoras
%
%  Created by Isaac Salvador Hernández Pompa on 2013-11-02.
%  Owners: Alejandro Cano e Isaac Pompa.
%
\documentclass[12pt,twocolumn]{report}

% Use utf-8 encoding for foreign characters
%\usepackage[utf8]{inputenc}

% Usar idioma, tipografía, entre otras cosas del español
\usepackage[spanish,activeacute]{babel}

% Para especificar pertenencia a instituciòn
\usepackage[affil-it]{authblk}

% 
\usepackage{fontspec}

% Tipografía TIMES NEW ROMAN
\setmainfont{Times New Roman}

% Setup for fullpage use
\usepackage{fullpage}

% Uncomment some of the following if you use the features
%
% Running Headers and footers
%\usepackage{fancyhdr}

% Multipart figures
%\usepackage{subfigure}

% More symbols
%\usepackage{amsmath}
%\usepackage{amssymb}
%\usepackage{latexsym}

% Surround parts of graphics with box
\usepackage{boxedminipage}

% Package for including code in the document
\usepackage{listings}

% If you want to generate a toc for each chapter (use with book)
\usepackage{minitoc}

% This is now the recommended way for checking for PDFLaTeX:
\usepackage{ifpdf}

%\newif\ifpdf
%\ifx\pdfoutput\undefined
%\pdffalse % we are not running PDFLaTeX
%\else
%\pdfoutput=1 % we are running PDFLaTeX
%\pdftrue
%\fi

\ifpdf
\usepackage[pdftex]{graphicx}
\else
\usepackage{graphicx}
\fi
\title{"Título, describe el contenido del trabajo"}
\author{Alejandro Cano,   Isaac Hernández}  % OJO no está respetando el doble espacio 
\affil{Universidad Nacional Autónoma de México}

\date{ }%15 de noviembre de 2013}

\begin{document}

\ifpdf
\DeclareGraphicsExtensions{.pdf, .jpg, .tif}
\else
\DeclareGraphicsExtensions{.eps, .jpg}
\fi

\maketitle


\begin{abstract}
\end{abstract}
% ++++++++ Aquí empieza el CONTENIDO del reporte ++++++++++
\section{Introducción}

La práctica me deja el conocimiento de cómo extraer elementos u objetos HTTP de capturas de paquetes para su posterior análisis.
También el uso de otras herramientas fuera del entorno UNIX/LINUX para análisis de paquetes, tal es el caso de TCPView. El análisis
del \texttt{md5sum} para verificar si un archivo ha sido modificado o no. Saber que es posible "emular" programas exclusivos de Windows
por medio de la herramienta \texttt{mono}.
El sistema operativo que se ocupó para realizar las diversas pruebas fue Windows XP.

\bibliographystyle{plain}
\bibliography{}
\end{document}
