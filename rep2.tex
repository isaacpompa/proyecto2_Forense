%
%  Reporte. Proyecto 2. Análisis forense de redes de computadoras
%
%  Created by Isaac Salvador Hernández Pompa on 2013-11-02.
%  Owners: Alejandro Cano e Isaac Pompa.
%
\documentclass[12pt,twocolumn]{report}

% Use utf-8 encoding for foreign characters
%\usepackage[utf8]{inputenc}

% Usar idioma, tipografía, entre otras cosas del español
\usepackage[spanish,activeacute]{babel}

% Para especificar pertenencia a instituciòn
\usepackage[affil-it]{authblk}

% 
\usepackage{fontspec}

% Tipografía TIMES NEW ROMAN
\setmainfont{Times New Roman}

% Setup for fullpage use
\usepackage{fullpage}

% Uncomment some of the following if you use the features
%
% Running Headers and footers
%\usepackage{fancyhdr}

% Multipart figures
%\usepackage{subfigure}

% More symbols
%\usepackage{amsmath}
%\usepackage{amssymb}
%\usepackage{latexsym}

% Surround parts of graphics with box
\usepackage{boxedminipage}

% Package for including code in the document
\usepackage{listings}

% If you want to generate a toc for each chapter (use with book)
\usepackage{minitoc}

% This is now the recommended way for checking for PDFLaTeX:
\usepackage{ifpdf}

%
\usepackage{ textcomp }

\usepackage{ amssymb }

%\newif\ifpdf
%\ifx\pdfoutput\undefined
%\pdffalse % we are not running PDFLaTeX
%\else
%\pdfoutput=1 % we are running PDFLaTeX
%\pdftrue
%\fi

\ifpdf
\usepackage[pdftex]{graphicx}
\else
\usepackage{graphicx}
\fi
\title{"Título, describe el contenido del trabajo"}
\author{Alejandro Cano,   Isaac Hernández}  % OJO no está respetando el doble espacio 
\affil{Universidad Nacional Autónoma de México}

\date{ }%15 de noviembre de 2013}

\begin{document}

\ifpdf
\DeclareGraphicsExtensions{.pdf, .jpg, .tif}
\else
\DeclareGraphicsExtensions{.eps, .jpg}
\fi

\maketitle


\begin{abstract}
\end{abstract}
% ++++++++ Aquí empieza el CONTENIDO del reporte ++++++++++
\section{Introducción}

El uso de teléfonos móviles sea ha incrementado en los últimos años, por eso es de suma importancia 
considerarlos dentro de los dispositivos candidatos a ser atacados a través de vulnerabilidades en su sistema 
operativo. El \textbf{Android Open Source Proyect} creado por la empresa \textbf{Google} representa alrededor 
del 70\% de los sistemas operativos para telefonía móvil seguido por el \textbf{iOS} de \textbf{Apple}.
El ataque del día cero o zero-day attack es uno de los más poderosos en el sentido de que se obtienen 
privilegios de superusuario en el sistema infectado, esto otorga al atacante un amplio espectro de facilidades 
para conocer el estado actual del sistema y la facilidad de alterarlo para su propio beneficio entre los que destacan
el robo de información confidencial como bases de datos, passwords, instalación y ejecución de malware. También se 
puede acceder a otros equipos (vía el equipo infectado) para ampliar la zona de infección. El zero-day attack 
tiene la característica de explotar las vulnerabilidades aún desconocidas por el fabricante de dicho sistema, 
de ahí proviene su nombre. Dentro de los dispositivos móviles, más específicamente en los Android, actualmente 
(En Julio de 2013) vio luz pública un exploit para ganar privilegios dentro del sistema. A esto se le conoce como 
un $"$Master Key$"$. Esto funciona como una llave maestra que permite al atacante acceder vía remota y con privilegios 
a todo el árbol de directorios del dispositivo, permite subir y bajar archivos o directorios completos, hacer uso del 
hardware como por ejemplo el micrófono y la(s) cámara(s). El modo de infección adoptado por los atacantes es por medio 
de un archivo \texttt{.APK} .

\section{Literatura}
Este ataque se clasifica como activo en el cual puede haber interrupción, modificación, suplantación y destrucción. Se 
cataloga como un ataque interno que se ejecuta desde la APK infectada debido a que no se rompen las firmas criptográficas 
de la aplicación lo que aparenta no tener código malicioso dentro de ella. La vulnerabilidad se podría remontar a las primeras
versiones de Android según TheHackerNews. Expertos en seguridad recomiendan sólo la instalación de aplicaciones provenientes de 
Google Play, de fuentes legítimas, verificar que el nombre del desarrollador sea válido y configurar el teléfono para que no 
instale aplicaciones de fuentes desconocidas.

Nos basaremos en la pila de vulnerabilidades y ataques para describir el ataque. Ubicándonos en la capa ocho de la pila de 
vulnerabilidades se usa ingeniería social para 
convencer al usuario de la necesidad de cierto tipo de aplicación para que su dispositivo pueda ejecutar (por ejemplo flash), 
mejorar el rendimiento del dispositivo, la necesidad de un $"$Antivirus$"$, etc. Después respectivo a la capa siete de la pila 
se ejecuta software malicioso a nivel de aplicación que realiza una conexión a un servidor previamente configurado por un puerto 
desconocido para que se obtenga acceso remotamente. Es un ataque silencioso que mientras la aplicación esté en ejecución se podrá
obtener acceso al sistema, cabe destacar que al momento de la conexión y una vez establecida no se necesita ningún tipo de 
autenticación que permitan la entrada al sistema. Este ataque también se encuentra en la capa cuatro de la pila, ya que, 
remotamente se pueden listar los servicios que el teléfono tiene actualmente levantados. De la capa uno y dos, podemos encontrar 
que se pueden realizar escaneos pasivamente, extracción de archivos como bases de datos, imágenes, videos, etc. Así como activar
el micrófono por periodos de tiempo prolongados y también hacer uso de la(s) cámara(s).

\section{Materiales y métodos}
\subsection{Requerimientos necesarios}
Para la reconstrucción del ataque en un ambiente controlado necesitamos la suite de metasploit(describir el proceso de 
instalación). Un teléfono móvil con las siguientes características:
\begin{enumerate}
\item Sistema operativo Android.
\item Contar con por lo menos una cámara.
\item Contar con conectividad Wi-Fi.
\item Tener por lo menos una aplicación que almacene en archivos locales su base de datos, por ejemplo WhatsApp.
\item Tener unas cuantas imágenes y/o videos.
\end{enumerate}

Además una máquina virtual con Debian 7 de 32 bits y un punto de acceso para trabajar sobre una LAN.

\subsubsection{Instalando Metasploit Framework}
En una terminal con permisos de superusuario escribir los siguientes comandos, actualización de repositorios:

\begin{lstlisting}
apt-get update
\end{lstlisting}


Ahora instalaremos las dependencias para \texttt{Metasploit Framework}:

\begin{lstlisting}
apt-get -y install
 build-essential
 libreadline-dev  libssl-dev
 libpq5 libpq-dev libreadline5
 libsqlite3-dev libpcap-dev
 openjdk-7-jre subversion
 git-core autoconf postgresql
 pgadmin3 curl zlib1g-dev
 libxml2-dev libxslt1-dev
 vncviewer libyaml-dev
 ruby1.9.3
\end{lstlisting}


A continuación instalaremos bibliotecas de Ruby de las cuales depende metasploit:

\begin{lstlisting}
gem install wirble sqlite3 bundler
\end{lstlisting}


\subsubsection{Instalando herramienta de escaneos Nmap}

\begin{lstlisting}
mkdir ~/Development
cd ~/Development
svn co https://svn.nmap.org/nmap
cd nmap
./configure
make
sudo make install
make clean
\end{lstlisting}


\subsubsection{Configurando el servidor de Postgre SQL}

Usamos el usuario de postgres para crear la base de datos que usaremos para metasploit

\begin{lstlisting}
su postres
\end{lstlisting}


Ahora crearemos la base de datos:

\begin{lstlisting}
createuser msf -P -S -R -D

createdb -O msf msf

exit
\end{lstlisting}


\subsubsection{Instalando el Framework de Metasploit}

Descargaremos los fuente desde Git para poder usar el comando \texttt{msfupdate} y estar siempre actualizados

\begin{lstlisting}
cd /opt

git clone
 https://github.com/rapid7/
metasploit-framework.git

cd metasploit-framework
\end{lstlisting}

\subsubsection{Instalando armitage}
\begin{lstlisting}
curl -# -o /tmp/armitage.tgz
 http://www.fastandeasyhack
ing.com/download/armitage
-latest.tgz

tar -xvzf /tmp/armitage.tgz
 -C /opt

ln -s /opt/armitage/armitage
 /usr/local/bin/armitage

ln -s /opt/armitage/teamserver
 /usr/local/bin/teamserver

echo java -jar /usr/local
/share/armitage/armitage.jar
 \$\* > /usr/local/share/
armitage/armitage

perl -pi -e 's/armitage.jar/
\/usr\/local\/share\/armitage
\/armitage.jar/g' /usr/local
/share/armitage/teamserver
\end{lstlisting}

Usaremos Bundler Gem para soportar apropiadamente las versiones de la Gema:

\begin{lstlisting}
bundle install

cd metasploit-framework
 bundle install
\end{lstlisting}


A continuación crearemos el archivo de la base de datos que contendrá los parámetros de configuración que usaremos para el 
framework:

\begin{lstlisting}
vim /opt/metasploit-framework/
database.yml
\end{lstlisting}


Copiar correctamente estas entradas YAML y asegúrate de proporcionar el password que asignaste al momento de la instalación de la 
base de datos:

\begin{lstlisting}
production:
   adapter: postgresql
   database: msf
   username: msf
   password: 
   host: 127.0.0.1
   port: 5432
   pool: 75
   timeout: 5
\end{lstlisting}

Crea la variable de ambiente para tanto Armitage como msfconsole corran en tu shell actual:
\begin{lstlisting}
sh -c echo export MSF_DATABASE_
CONFIG=/opt/metasploit-framework/
database.yml >> /etc/profile
\end{lstlisting}

\begin{lstlisting}
cd /opt/metasploit-framework/external/
pcaprub

sudo ruby extconf.rb && sudo make &&
 sudo make install
\end{lstlisting}

Probamos que funcione:
\begin{lstlisting}
msfconsole
\end{lstlisting}

\bibliographystyle{plain}
\bibliography{}
\end{document}
